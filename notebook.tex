
% Default to the notebook output style

    


% Inherit from the specified cell style.




    
\documentclass[11pt]{article}

    
    
    \usepackage[T1]{fontenc}
    % Nicer default font (+ math font) than Computer Modern for most use cases
    \usepackage{mathpazo}

    % Basic figure setup, for now with no caption control since it's done
    % automatically by Pandoc (which extracts ![](path) syntax from Markdown).
    \usepackage{graphicx}
    % We will generate all images so they have a width \maxwidth. This means
    % that they will get their normal width if they fit onto the page, but
    % are scaled down if they would overflow the margins.
    \makeatletter
    \def\maxwidth{\ifdim\Gin@nat@width>\linewidth\linewidth
    \else\Gin@nat@width\fi}
    \makeatother
    \let\Oldincludegraphics\includegraphics
    % Set max figure width to be 80% of text width, for now hardcoded.
    \renewcommand{\includegraphics}[1]{\Oldincludegraphics[width=.8\maxwidth]{#1}}
    % Ensure that by default, figures have no caption (until we provide a
    % proper Figure object with a Caption API and a way to capture that
    % in the conversion process - todo).
    \usepackage{caption}
    \DeclareCaptionLabelFormat{nolabel}{}
    \captionsetup{labelformat=nolabel}

    \usepackage{adjustbox} % Used to constrain images to a maximum size 
    \usepackage{xcolor} % Allow colors to be defined
    \usepackage{enumerate} % Needed for markdown enumerations to work
    \usepackage{geometry} % Used to adjust the document margins
    \usepackage{amsmath} % Equations
    \usepackage{amssymb} % Equations
    \usepackage{textcomp} % defines textquotesingle
    % Hack from http://tex.stackexchange.com/a/47451/13684:
    \AtBeginDocument{%
        \def\PYZsq{\textquotesingle}% Upright quotes in Pygmentized code
    }
    \usepackage{upquote} % Upright quotes for verbatim code
    \usepackage{eurosym} % defines \euro
    \usepackage[mathletters]{ucs} % Extended unicode (utf-8) support
    \usepackage[utf8x]{inputenc} % Allow utf-8 characters in the tex document
    \usepackage{fancyvrb} % verbatim replacement that allows latex
    \usepackage{grffile} % extends the file name processing of package graphics 
                         % to support a larger range 
    % The hyperref package gives us a pdf with properly built
    % internal navigation ('pdf bookmarks' for the table of contents,
    % internal cross-reference links, web links for URLs, etc.)
    \usepackage{hyperref}
    \usepackage{longtable} % longtable support required by pandoc >1.10
    \usepackage{booktabs}  % table support for pandoc > 1.12.2
    \usepackage[inline]{enumitem} % IRkernel/repr support (it uses the enumerate* environment)
    \usepackage[normalem]{ulem} % ulem is needed to support strikethroughs (\sout)
                                % normalem makes italics be italics, not underlines
    

    
    
    % Colors for the hyperref package
    \definecolor{urlcolor}{rgb}{0,.145,.698}
    \definecolor{linkcolor}{rgb}{.71,0.21,0.01}
    \definecolor{citecolor}{rgb}{.12,.54,.11}

    % ANSI colors
    \definecolor{ansi-black}{HTML}{3E424D}
    \definecolor{ansi-black-intense}{HTML}{282C36}
    \definecolor{ansi-red}{HTML}{E75C58}
    \definecolor{ansi-red-intense}{HTML}{B22B31}
    \definecolor{ansi-green}{HTML}{00A250}
    \definecolor{ansi-green-intense}{HTML}{007427}
    \definecolor{ansi-yellow}{HTML}{DDB62B}
    \definecolor{ansi-yellow-intense}{HTML}{B27D12}
    \definecolor{ansi-blue}{HTML}{208FFB}
    \definecolor{ansi-blue-intense}{HTML}{0065CA}
    \definecolor{ansi-magenta}{HTML}{D160C4}
    \definecolor{ansi-magenta-intense}{HTML}{A03196}
    \definecolor{ansi-cyan}{HTML}{60C6C8}
    \definecolor{ansi-cyan-intense}{HTML}{258F8F}
    \definecolor{ansi-white}{HTML}{C5C1B4}
    \definecolor{ansi-white-intense}{HTML}{A1A6B2}

    % commands and environments needed by pandoc snippets
    % extracted from the output of `pandoc -s`
    \providecommand{\tightlist}{%
      \setlength{\itemsep}{0pt}\setlength{\parskip}{0pt}}
    \DefineVerbatimEnvironment{Highlighting}{Verbatim}{commandchars=\\\{\}}
    % Add ',fontsize=\small' for more characters per line
    \newenvironment{Shaded}{}{}
    \newcommand{\KeywordTok}[1]{\textcolor[rgb]{0.00,0.44,0.13}{\textbf{{#1}}}}
    \newcommand{\DataTypeTok}[1]{\textcolor[rgb]{0.56,0.13,0.00}{{#1}}}
    \newcommand{\DecValTok}[1]{\textcolor[rgb]{0.25,0.63,0.44}{{#1}}}
    \newcommand{\BaseNTok}[1]{\textcolor[rgb]{0.25,0.63,0.44}{{#1}}}
    \newcommand{\FloatTok}[1]{\textcolor[rgb]{0.25,0.63,0.44}{{#1}}}
    \newcommand{\CharTok}[1]{\textcolor[rgb]{0.25,0.44,0.63}{{#1}}}
    \newcommand{\StringTok}[1]{\textcolor[rgb]{0.25,0.44,0.63}{{#1}}}
    \newcommand{\CommentTok}[1]{\textcolor[rgb]{0.38,0.63,0.69}{\textit{{#1}}}}
    \newcommand{\OtherTok}[1]{\textcolor[rgb]{0.00,0.44,0.13}{{#1}}}
    \newcommand{\AlertTok}[1]{\textcolor[rgb]{1.00,0.00,0.00}{\textbf{{#1}}}}
    \newcommand{\FunctionTok}[1]{\textcolor[rgb]{0.02,0.16,0.49}{{#1}}}
    \newcommand{\RegionMarkerTok}[1]{{#1}}
    \newcommand{\ErrorTok}[1]{\textcolor[rgb]{1.00,0.00,0.00}{\textbf{{#1}}}}
    \newcommand{\NormalTok}[1]{{#1}}
    
    % Additional commands for more recent versions of Pandoc
    \newcommand{\ConstantTok}[1]{\textcolor[rgb]{0.53,0.00,0.00}{{#1}}}
    \newcommand{\SpecialCharTok}[1]{\textcolor[rgb]{0.25,0.44,0.63}{{#1}}}
    \newcommand{\VerbatimStringTok}[1]{\textcolor[rgb]{0.25,0.44,0.63}{{#1}}}
    \newcommand{\SpecialStringTok}[1]{\textcolor[rgb]{0.73,0.40,0.53}{{#1}}}
    \newcommand{\ImportTok}[1]{{#1}}
    \newcommand{\DocumentationTok}[1]{\textcolor[rgb]{0.73,0.13,0.13}{\textit{{#1}}}}
    \newcommand{\AnnotationTok}[1]{\textcolor[rgb]{0.38,0.63,0.69}{\textbf{\textit{{#1}}}}}
    \newcommand{\CommentVarTok}[1]{\textcolor[rgb]{0.38,0.63,0.69}{\textbf{\textit{{#1}}}}}
    \newcommand{\VariableTok}[1]{\textcolor[rgb]{0.10,0.09,0.49}{{#1}}}
    \newcommand{\ControlFlowTok}[1]{\textcolor[rgb]{0.00,0.44,0.13}{\textbf{{#1}}}}
    \newcommand{\OperatorTok}[1]{\textcolor[rgb]{0.40,0.40,0.40}{{#1}}}
    \newcommand{\BuiltInTok}[1]{{#1}}
    \newcommand{\ExtensionTok}[1]{{#1}}
    \newcommand{\PreprocessorTok}[1]{\textcolor[rgb]{0.74,0.48,0.00}{{#1}}}
    \newcommand{\AttributeTok}[1]{\textcolor[rgb]{0.49,0.56,0.16}{{#1}}}
    \newcommand{\InformationTok}[1]{\textcolor[rgb]{0.38,0.63,0.69}{\textbf{\textit{{#1}}}}}
    \newcommand{\WarningTok}[1]{\textcolor[rgb]{0.38,0.63,0.69}{\textbf{\textit{{#1}}}}}
    
    
    % Define a nice break command that doesn't care if a line doesn't already
    % exist.
    \def\br{\hspace*{\fill} \\* }
    % Math Jax compatability definitions
    \def\gt{>}
    \def\lt{<}
    % Document parameters
    \title{1. Load and Visualize Data}
    
    
    

    % Pygments definitions
    
\makeatletter
\def\PY@reset{\let\PY@it=\relax \let\PY@bf=\relax%
    \let\PY@ul=\relax \let\PY@tc=\relax%
    \let\PY@bc=\relax \let\PY@ff=\relax}
\def\PY@tok#1{\csname PY@tok@#1\endcsname}
\def\PY@toks#1+{\ifx\relax#1\empty\else%
    \PY@tok{#1}\expandafter\PY@toks\fi}
\def\PY@do#1{\PY@bc{\PY@tc{\PY@ul{%
    \PY@it{\PY@bf{\PY@ff{#1}}}}}}}
\def\PY#1#2{\PY@reset\PY@toks#1+\relax+\PY@do{#2}}

\expandafter\def\csname PY@tok@w\endcsname{\def\PY@tc##1{\textcolor[rgb]{0.73,0.73,0.73}{##1}}}
\expandafter\def\csname PY@tok@c\endcsname{\let\PY@it=\textit\def\PY@tc##1{\textcolor[rgb]{0.25,0.50,0.50}{##1}}}
\expandafter\def\csname PY@tok@cp\endcsname{\def\PY@tc##1{\textcolor[rgb]{0.74,0.48,0.00}{##1}}}
\expandafter\def\csname PY@tok@k\endcsname{\let\PY@bf=\textbf\def\PY@tc##1{\textcolor[rgb]{0.00,0.50,0.00}{##1}}}
\expandafter\def\csname PY@tok@kp\endcsname{\def\PY@tc##1{\textcolor[rgb]{0.00,0.50,0.00}{##1}}}
\expandafter\def\csname PY@tok@kt\endcsname{\def\PY@tc##1{\textcolor[rgb]{0.69,0.00,0.25}{##1}}}
\expandafter\def\csname PY@tok@o\endcsname{\def\PY@tc##1{\textcolor[rgb]{0.40,0.40,0.40}{##1}}}
\expandafter\def\csname PY@tok@ow\endcsname{\let\PY@bf=\textbf\def\PY@tc##1{\textcolor[rgb]{0.67,0.13,1.00}{##1}}}
\expandafter\def\csname PY@tok@nb\endcsname{\def\PY@tc##1{\textcolor[rgb]{0.00,0.50,0.00}{##1}}}
\expandafter\def\csname PY@tok@nf\endcsname{\def\PY@tc##1{\textcolor[rgb]{0.00,0.00,1.00}{##1}}}
\expandafter\def\csname PY@tok@nc\endcsname{\let\PY@bf=\textbf\def\PY@tc##1{\textcolor[rgb]{0.00,0.00,1.00}{##1}}}
\expandafter\def\csname PY@tok@nn\endcsname{\let\PY@bf=\textbf\def\PY@tc##1{\textcolor[rgb]{0.00,0.00,1.00}{##1}}}
\expandafter\def\csname PY@tok@ne\endcsname{\let\PY@bf=\textbf\def\PY@tc##1{\textcolor[rgb]{0.82,0.25,0.23}{##1}}}
\expandafter\def\csname PY@tok@nv\endcsname{\def\PY@tc##1{\textcolor[rgb]{0.10,0.09,0.49}{##1}}}
\expandafter\def\csname PY@tok@no\endcsname{\def\PY@tc##1{\textcolor[rgb]{0.53,0.00,0.00}{##1}}}
\expandafter\def\csname PY@tok@nl\endcsname{\def\PY@tc##1{\textcolor[rgb]{0.63,0.63,0.00}{##1}}}
\expandafter\def\csname PY@tok@ni\endcsname{\let\PY@bf=\textbf\def\PY@tc##1{\textcolor[rgb]{0.60,0.60,0.60}{##1}}}
\expandafter\def\csname PY@tok@na\endcsname{\def\PY@tc##1{\textcolor[rgb]{0.49,0.56,0.16}{##1}}}
\expandafter\def\csname PY@tok@nt\endcsname{\let\PY@bf=\textbf\def\PY@tc##1{\textcolor[rgb]{0.00,0.50,0.00}{##1}}}
\expandafter\def\csname PY@tok@nd\endcsname{\def\PY@tc##1{\textcolor[rgb]{0.67,0.13,1.00}{##1}}}
\expandafter\def\csname PY@tok@s\endcsname{\def\PY@tc##1{\textcolor[rgb]{0.73,0.13,0.13}{##1}}}
\expandafter\def\csname PY@tok@sd\endcsname{\let\PY@it=\textit\def\PY@tc##1{\textcolor[rgb]{0.73,0.13,0.13}{##1}}}
\expandafter\def\csname PY@tok@si\endcsname{\let\PY@bf=\textbf\def\PY@tc##1{\textcolor[rgb]{0.73,0.40,0.53}{##1}}}
\expandafter\def\csname PY@tok@se\endcsname{\let\PY@bf=\textbf\def\PY@tc##1{\textcolor[rgb]{0.73,0.40,0.13}{##1}}}
\expandafter\def\csname PY@tok@sr\endcsname{\def\PY@tc##1{\textcolor[rgb]{0.73,0.40,0.53}{##1}}}
\expandafter\def\csname PY@tok@ss\endcsname{\def\PY@tc##1{\textcolor[rgb]{0.10,0.09,0.49}{##1}}}
\expandafter\def\csname PY@tok@sx\endcsname{\def\PY@tc##1{\textcolor[rgb]{0.00,0.50,0.00}{##1}}}
\expandafter\def\csname PY@tok@m\endcsname{\def\PY@tc##1{\textcolor[rgb]{0.40,0.40,0.40}{##1}}}
\expandafter\def\csname PY@tok@gh\endcsname{\let\PY@bf=\textbf\def\PY@tc##1{\textcolor[rgb]{0.00,0.00,0.50}{##1}}}
\expandafter\def\csname PY@tok@gu\endcsname{\let\PY@bf=\textbf\def\PY@tc##1{\textcolor[rgb]{0.50,0.00,0.50}{##1}}}
\expandafter\def\csname PY@tok@gd\endcsname{\def\PY@tc##1{\textcolor[rgb]{0.63,0.00,0.00}{##1}}}
\expandafter\def\csname PY@tok@gi\endcsname{\def\PY@tc##1{\textcolor[rgb]{0.00,0.63,0.00}{##1}}}
\expandafter\def\csname PY@tok@gr\endcsname{\def\PY@tc##1{\textcolor[rgb]{1.00,0.00,0.00}{##1}}}
\expandafter\def\csname PY@tok@ge\endcsname{\let\PY@it=\textit}
\expandafter\def\csname PY@tok@gs\endcsname{\let\PY@bf=\textbf}
\expandafter\def\csname PY@tok@gp\endcsname{\let\PY@bf=\textbf\def\PY@tc##1{\textcolor[rgb]{0.00,0.00,0.50}{##1}}}
\expandafter\def\csname PY@tok@go\endcsname{\def\PY@tc##1{\textcolor[rgb]{0.53,0.53,0.53}{##1}}}
\expandafter\def\csname PY@tok@gt\endcsname{\def\PY@tc##1{\textcolor[rgb]{0.00,0.27,0.87}{##1}}}
\expandafter\def\csname PY@tok@err\endcsname{\def\PY@bc##1{\setlength{\fboxsep}{0pt}\fcolorbox[rgb]{1.00,0.00,0.00}{1,1,1}{\strut ##1}}}
\expandafter\def\csname PY@tok@kc\endcsname{\let\PY@bf=\textbf\def\PY@tc##1{\textcolor[rgb]{0.00,0.50,0.00}{##1}}}
\expandafter\def\csname PY@tok@kd\endcsname{\let\PY@bf=\textbf\def\PY@tc##1{\textcolor[rgb]{0.00,0.50,0.00}{##1}}}
\expandafter\def\csname PY@tok@kn\endcsname{\let\PY@bf=\textbf\def\PY@tc##1{\textcolor[rgb]{0.00,0.50,0.00}{##1}}}
\expandafter\def\csname PY@tok@kr\endcsname{\let\PY@bf=\textbf\def\PY@tc##1{\textcolor[rgb]{0.00,0.50,0.00}{##1}}}
\expandafter\def\csname PY@tok@bp\endcsname{\def\PY@tc##1{\textcolor[rgb]{0.00,0.50,0.00}{##1}}}
\expandafter\def\csname PY@tok@fm\endcsname{\def\PY@tc##1{\textcolor[rgb]{0.00,0.00,1.00}{##1}}}
\expandafter\def\csname PY@tok@vc\endcsname{\def\PY@tc##1{\textcolor[rgb]{0.10,0.09,0.49}{##1}}}
\expandafter\def\csname PY@tok@vg\endcsname{\def\PY@tc##1{\textcolor[rgb]{0.10,0.09,0.49}{##1}}}
\expandafter\def\csname PY@tok@vi\endcsname{\def\PY@tc##1{\textcolor[rgb]{0.10,0.09,0.49}{##1}}}
\expandafter\def\csname PY@tok@vm\endcsname{\def\PY@tc##1{\textcolor[rgb]{0.10,0.09,0.49}{##1}}}
\expandafter\def\csname PY@tok@sa\endcsname{\def\PY@tc##1{\textcolor[rgb]{0.73,0.13,0.13}{##1}}}
\expandafter\def\csname PY@tok@sb\endcsname{\def\PY@tc##1{\textcolor[rgb]{0.73,0.13,0.13}{##1}}}
\expandafter\def\csname PY@tok@sc\endcsname{\def\PY@tc##1{\textcolor[rgb]{0.73,0.13,0.13}{##1}}}
\expandafter\def\csname PY@tok@dl\endcsname{\def\PY@tc##1{\textcolor[rgb]{0.73,0.13,0.13}{##1}}}
\expandafter\def\csname PY@tok@s2\endcsname{\def\PY@tc##1{\textcolor[rgb]{0.73,0.13,0.13}{##1}}}
\expandafter\def\csname PY@tok@sh\endcsname{\def\PY@tc##1{\textcolor[rgb]{0.73,0.13,0.13}{##1}}}
\expandafter\def\csname PY@tok@s1\endcsname{\def\PY@tc##1{\textcolor[rgb]{0.73,0.13,0.13}{##1}}}
\expandafter\def\csname PY@tok@mb\endcsname{\def\PY@tc##1{\textcolor[rgb]{0.40,0.40,0.40}{##1}}}
\expandafter\def\csname PY@tok@mf\endcsname{\def\PY@tc##1{\textcolor[rgb]{0.40,0.40,0.40}{##1}}}
\expandafter\def\csname PY@tok@mh\endcsname{\def\PY@tc##1{\textcolor[rgb]{0.40,0.40,0.40}{##1}}}
\expandafter\def\csname PY@tok@mi\endcsname{\def\PY@tc##1{\textcolor[rgb]{0.40,0.40,0.40}{##1}}}
\expandafter\def\csname PY@tok@il\endcsname{\def\PY@tc##1{\textcolor[rgb]{0.40,0.40,0.40}{##1}}}
\expandafter\def\csname PY@tok@mo\endcsname{\def\PY@tc##1{\textcolor[rgb]{0.40,0.40,0.40}{##1}}}
\expandafter\def\csname PY@tok@ch\endcsname{\let\PY@it=\textit\def\PY@tc##1{\textcolor[rgb]{0.25,0.50,0.50}{##1}}}
\expandafter\def\csname PY@tok@cm\endcsname{\let\PY@it=\textit\def\PY@tc##1{\textcolor[rgb]{0.25,0.50,0.50}{##1}}}
\expandafter\def\csname PY@tok@cpf\endcsname{\let\PY@it=\textit\def\PY@tc##1{\textcolor[rgb]{0.25,0.50,0.50}{##1}}}
\expandafter\def\csname PY@tok@c1\endcsname{\let\PY@it=\textit\def\PY@tc##1{\textcolor[rgb]{0.25,0.50,0.50}{##1}}}
\expandafter\def\csname PY@tok@cs\endcsname{\let\PY@it=\textit\def\PY@tc##1{\textcolor[rgb]{0.25,0.50,0.50}{##1}}}

\def\PYZbs{\char`\\}
\def\PYZus{\char`\_}
\def\PYZob{\char`\{}
\def\PYZcb{\char`\}}
\def\PYZca{\char`\^}
\def\PYZam{\char`\&}
\def\PYZlt{\char`\<}
\def\PYZgt{\char`\>}
\def\PYZsh{\char`\#}
\def\PYZpc{\char`\%}
\def\PYZdl{\char`\$}
\def\PYZhy{\char`\-}
\def\PYZsq{\char`\'}
\def\PYZdq{\char`\"}
\def\PYZti{\char`\~}
% for compatibility with earlier versions
\def\PYZat{@}
\def\PYZlb{[}
\def\PYZrb{]}
\makeatother


    % Exact colors from NB
    \definecolor{incolor}{rgb}{0.0, 0.0, 0.5}
    \definecolor{outcolor}{rgb}{0.545, 0.0, 0.0}



    
    % Prevent overflowing lines due to hard-to-break entities
    \sloppy 
    % Setup hyperref package
    \hypersetup{
      breaklinks=true,  % so long urls are correctly broken across lines
      colorlinks=true,
      urlcolor=urlcolor,
      linkcolor=linkcolor,
      citecolor=citecolor,
      }
    % Slightly bigger margins than the latex defaults
    
    \geometry{verbose,tmargin=1in,bmargin=1in,lmargin=1in,rmargin=1in}
    
    

    \begin{document}
    
    
    \maketitle
    
    

    
    \section{Facial Keypoint Detection}\label{facial-keypoint-detection}

This project will be all about defining and training a convolutional
neural network to perform facial keypoint detection, and using computer
vision techniques to transform images of faces. The first step in any
challenge like this will be to load and visualize the data you'll be
working with.

Let's take a look at some examples of images and corresponding facial
keypoints.

Facial keypoints (also called facial landmarks) are the small magenta
dots shown on each of the faces in the image above. In each training and
test image, there is a single face and \textbf{68 keypoints, with
coordinates (x, y), for that face}. These keypoints mark important areas
of the face: the eyes, corners of the mouth, the nose, etc. These
keypoints are relevant for a variety of tasks, such as face filters,
emotion recognition, pose recognition, and so on. Here they are,
numbered, and you can see that specific ranges of points match different
portions of the face.

\begin{center}\rule{0.5\linewidth}{\linethickness}\end{center}

    \subsection{Load and Visualize Data}\label{load-and-visualize-data}

The first step in working with any dataset is to become familiar with
your data; you'll need to load in the images of faces and their
keypoints and visualize them! This set of image data has been extracted
from the \href{https://www.cs.tau.ac.il/~wolf/ytfaces/}{YouTube Faces
Dataset}, which includes videos of people in YouTube videos. These
videos have been fed through some processing steps and turned into sets
of image frames containing one face and the associated keypoints.

\paragraph{Training and Testing Data}\label{training-and-testing-data}

This facial keypoints dataset consists of 5770 color images. All of
these images are separated into either a training or a test set of data.

\begin{itemize}
\tightlist
\item
  3462 of these images are training images, for you to use as you create
  a model to predict keypoints.
\item
  2308 are test images, which will be used to test the accuracy of your
  model.
\end{itemize}

The information about the images and keypoints in this dataset are
summarized in CSV files, which we can read in using \texttt{pandas}.
Let's read the training CSV and get the annotations in an (N, 2) array
where N is the number of keypoints and 2 is the dimension of the
keypoint coordinates (x, y).

\begin{center}\rule{0.5\linewidth}{\linethickness}\end{center}

    \begin{Verbatim}[commandchars=\\\{\}]
{\color{incolor}In [{\color{incolor}1}]:} \PY{c+c1}{\PYZsh{} import the required libraries}
        \PY{k+kn}{import} \PY{n+nn}{glob}
        \PY{k+kn}{import} \PY{n+nn}{os}
        \PY{k+kn}{import} \PY{n+nn}{numpy} \PY{k}{as} \PY{n+nn}{np}
        \PY{k+kn}{import} \PY{n+nn}{pandas} \PY{k}{as} \PY{n+nn}{pd}
        \PY{k+kn}{import} \PY{n+nn}{matplotlib}\PY{n+nn}{.}\PY{n+nn}{pyplot} \PY{k}{as} \PY{n+nn}{plt}
        \PY{k+kn}{import} \PY{n+nn}{matplotlib}\PY{n+nn}{.}\PY{n+nn}{image} \PY{k}{as} \PY{n+nn}{mpimg}
        
        \PY{k+kn}{import} \PY{n+nn}{cv2}
\end{Verbatim}


    \begin{Verbatim}[commandchars=\\\{\}]
{\color{incolor}In [{\color{incolor}2}]:} \PY{n}{key\PYZus{}pts\PYZus{}frame} \PY{o}{=} \PY{n}{pd}\PY{o}{.}\PY{n}{read\PYZus{}csv}\PY{p}{(}\PY{l+s+s1}{\PYZsq{}}\PY{l+s+s1}{data/training\PYZus{}frames\PYZus{}keypoints.csv}\PY{l+s+s1}{\PYZsq{}}\PY{p}{)}
        
        \PY{n}{n} \PY{o}{=} \PY{l+m+mi}{0}
        \PY{n}{image\PYZus{}name} \PY{o}{=} \PY{n}{key\PYZus{}pts\PYZus{}frame}\PY{o}{.}\PY{n}{iloc}\PY{p}{[}\PY{n}{n}\PY{p}{,} \PY{l+m+mi}{0}\PY{p}{]}
        \PY{n}{key\PYZus{}pts} \PY{o}{=} \PY{n}{key\PYZus{}pts\PYZus{}frame}\PY{o}{.}\PY{n}{iloc}\PY{p}{[}\PY{n}{n}\PY{p}{,} \PY{l+m+mi}{1}\PY{p}{:}\PY{p}{]}\PY{o}{.}\PY{n}{as\PYZus{}matrix}\PY{p}{(}\PY{p}{)}
        \PY{n}{key\PYZus{}pts} \PY{o}{=} \PY{n}{key\PYZus{}pts}\PY{o}{.}\PY{n}{astype}\PY{p}{(}\PY{l+s+s1}{\PYZsq{}}\PY{l+s+s1}{float}\PY{l+s+s1}{\PYZsq{}}\PY{p}{)}\PY{o}{.}\PY{n}{reshape}\PY{p}{(}\PY{o}{\PYZhy{}}\PY{l+m+mi}{1}\PY{p}{,} \PY{l+m+mi}{2}\PY{p}{)}
        
        \PY{n+nb}{print}\PY{p}{(}\PY{l+s+s1}{\PYZsq{}}\PY{l+s+s1}{Image name: }\PY{l+s+s1}{\PYZsq{}}\PY{p}{,} \PY{n}{image\PYZus{}name}\PY{p}{)}
        \PY{n+nb}{print}\PY{p}{(}\PY{l+s+s1}{\PYZsq{}}\PY{l+s+s1}{Landmarks shape: }\PY{l+s+s1}{\PYZsq{}}\PY{p}{,} \PY{n}{key\PYZus{}pts}\PY{o}{.}\PY{n}{shape}\PY{p}{)}
        \PY{n+nb}{print}\PY{p}{(}\PY{l+s+s1}{\PYZsq{}}\PY{l+s+s1}{First 4 key pts: }\PY{l+s+si}{\PYZob{}\PYZcb{}}\PY{l+s+s1}{\PYZsq{}}\PY{o}{.}\PY{n}{format}\PY{p}{(}\PY{n}{key\PYZus{}pts}\PY{p}{[}\PY{p}{:}\PY{l+m+mi}{4}\PY{p}{]}\PY{p}{)}\PY{p}{)}
\end{Verbatim}


    \begin{Verbatim}[commandchars=\\\{\}]
Image name:  Luis\_Fonsi\_21.jpg
Landmarks shape:  (68, 2)
First 4 key pts: [[ 45.  98.]
 [ 47. 106.]
 [ 49. 110.]
 [ 53. 119.]]

    \end{Verbatim}

    \begin{Verbatim}[commandchars=\\\{\}]
{\color{incolor}In [{\color{incolor}3}]:} \PY{c+c1}{\PYZsh{} print out some stats about the data}
        \PY{n+nb}{print}\PY{p}{(}\PY{l+s+s1}{\PYZsq{}}\PY{l+s+s1}{Number of images: }\PY{l+s+s1}{\PYZsq{}}\PY{p}{,} \PY{n}{key\PYZus{}pts\PYZus{}frame}\PY{o}{.}\PY{n}{shape}\PY{p}{[}\PY{l+m+mi}{0}\PY{p}{]}\PY{p}{)}
\end{Verbatim}


    \begin{Verbatim}[commandchars=\\\{\}]
Number of images:  3462

    \end{Verbatim}

    \subsection{Look at some images}\label{look-at-some-images}

Below, is a function \texttt{show\_keypoints} that takes in an image and
keypoints and displays them. As you look at this data, \textbf{note that
these images are not all of the same size}, and neither are the faces!
To eventually train a neural network on these images, we'll need to
standardize their shape.

    \begin{Verbatim}[commandchars=\\\{\}]
{\color{incolor}In [{\color{incolor}4}]:} \PY{k}{def} \PY{n+nf}{show\PYZus{}keypoints}\PY{p}{(}\PY{n}{image}\PY{p}{,} \PY{n}{key\PYZus{}pts}\PY{p}{)}\PY{p}{:}
            \PY{l+s+sd}{\PYZdq{}\PYZdq{}\PYZdq{}Show image with keypoints\PYZdq{}\PYZdq{}\PYZdq{}}
            \PY{n}{plt}\PY{o}{.}\PY{n}{imshow}\PY{p}{(}\PY{n}{image}\PY{p}{)}
            \PY{n}{plt}\PY{o}{.}\PY{n}{scatter}\PY{p}{(}\PY{n}{key\PYZus{}pts}\PY{p}{[}\PY{p}{:}\PY{p}{,} \PY{l+m+mi}{0}\PY{p}{]}\PY{p}{,} \PY{n}{key\PYZus{}pts}\PY{p}{[}\PY{p}{:}\PY{p}{,} \PY{l+m+mi}{1}\PY{p}{]}\PY{p}{,} \PY{n}{s}\PY{o}{=}\PY{l+m+mi}{20}\PY{p}{,} \PY{n}{marker}\PY{o}{=}\PY{l+s+s1}{\PYZsq{}}\PY{l+s+s1}{.}\PY{l+s+s1}{\PYZsq{}}\PY{p}{,} \PY{n}{c}\PY{o}{=}\PY{l+s+s1}{\PYZsq{}}\PY{l+s+s1}{m}\PY{l+s+s1}{\PYZsq{}}\PY{p}{)}
\end{Verbatim}


    \begin{Verbatim}[commandchars=\\\{\}]
{\color{incolor}In [{\color{incolor}5}]:} \PY{c+c1}{\PYZsh{} Display a few different types of images by changing the index n}
        
        \PY{c+c1}{\PYZsh{} select an image by index in our data frame}
        \PY{n}{n} \PY{o}{=} \PY{l+m+mi}{0}
        \PY{n}{image\PYZus{}name} \PY{o}{=} \PY{n}{key\PYZus{}pts\PYZus{}frame}\PY{o}{.}\PY{n}{iloc}\PY{p}{[}\PY{n}{n}\PY{p}{,} \PY{l+m+mi}{0}\PY{p}{]}
        \PY{n}{key\PYZus{}pts} \PY{o}{=} \PY{n}{key\PYZus{}pts\PYZus{}frame}\PY{o}{.}\PY{n}{iloc}\PY{p}{[}\PY{n}{n}\PY{p}{,} \PY{l+m+mi}{1}\PY{p}{:}\PY{p}{]}\PY{o}{.}\PY{n}{as\PYZus{}matrix}\PY{p}{(}\PY{p}{)}
        \PY{n}{key\PYZus{}pts} \PY{o}{=} \PY{n}{key\PYZus{}pts}\PY{o}{.}\PY{n}{astype}\PY{p}{(}\PY{l+s+s1}{\PYZsq{}}\PY{l+s+s1}{float}\PY{l+s+s1}{\PYZsq{}}\PY{p}{)}\PY{o}{.}\PY{n}{reshape}\PY{p}{(}\PY{o}{\PYZhy{}}\PY{l+m+mi}{1}\PY{p}{,} \PY{l+m+mi}{2}\PY{p}{)}
        
        \PY{n}{plt}\PY{o}{.}\PY{n}{figure}\PY{p}{(}\PY{n}{figsize}\PY{o}{=}\PY{p}{(}\PY{l+m+mi}{5}\PY{p}{,} \PY{l+m+mi}{5}\PY{p}{)}\PY{p}{)}
        \PY{n}{show\PYZus{}keypoints}\PY{p}{(}\PY{n}{mpimg}\PY{o}{.}\PY{n}{imread}\PY{p}{(}\PY{n}{os}\PY{o}{.}\PY{n}{path}\PY{o}{.}\PY{n}{join}\PY{p}{(}\PY{l+s+s1}{\PYZsq{}}\PY{l+s+s1}{data/training/}\PY{l+s+s1}{\PYZsq{}}\PY{p}{,} \PY{n}{image\PYZus{}name}\PY{p}{)}\PY{p}{)}\PY{p}{,} \PY{n}{key\PYZus{}pts}\PY{p}{)}
        \PY{n}{plt}\PY{o}{.}\PY{n}{show}\PY{p}{(}\PY{p}{)}
\end{Verbatim}


    \begin{center}
    \adjustimage{max size={0.9\linewidth}{0.9\paperheight}}{output_7_0.png}
    \end{center}
    { \hspace*{\fill} \\}
    
    \subsection{Dataset class and
Transformations}\label{dataset-class-and-transformations}

To prepare our data for training, we'll be using PyTorch's Dataset
class. Much of this this code is a modified version of what can be found
in the
\href{http://pytorch.org/tutorials/beginner/data_loading_tutorial.html}{PyTorch
data loading tutorial}.

\paragraph{Dataset class}\label{dataset-class}

\texttt{torch.utils.data.Dataset} is an abstract class representing a
dataset. This class will allow us to load batches of image/keypoint
data, and uniformly apply transformations to our data, such as rescaling
and normalizing images for training a neural network.

Your custom dataset should inherit \texttt{Dataset} and override the
following methods:

\begin{itemize}
\tightlist
\item
  \texttt{\_\_len\_\_} so that \texttt{len(dataset)} returns the size of
  the dataset.
\item
  \texttt{\_\_getitem\_\_} to support the indexing such that
  \texttt{dataset{[}i{]}} can be used to get the i-th sample of
  image/keypoint data.
\end{itemize}

Let's create a dataset class for our face keypoints dataset. We will
read the CSV file in \texttt{\_\_init\_\_} but leave the reading of
images to \texttt{\_\_getitem\_\_}. This is memory efficient because all
the images are not stored in the memory at once but read as required.

A sample of our dataset will be a dictionary
\texttt{\{\textquotesingle{}image\textquotesingle{}:\ image,\ \textquotesingle{}keypoints\textquotesingle{}:\ key\_pts\}}.
Our dataset will take an optional argument \texttt{transform} so that
any required processing can be applied on the sample. We will see the
usefulness of \texttt{transform} in the next section.

    \begin{Verbatim}[commandchars=\\\{\}]
{\color{incolor}In [{\color{incolor}6}]:} \PY{k+kn}{from} \PY{n+nn}{torch}\PY{n+nn}{.}\PY{n+nn}{utils}\PY{n+nn}{.}\PY{n+nn}{data} \PY{k}{import} \PY{n}{Dataset}\PY{p}{,} \PY{n}{DataLoader}
        
        \PY{k}{class} \PY{n+nc}{FacialKeypointsDataset}\PY{p}{(}\PY{n}{Dataset}\PY{p}{)}\PY{p}{:}
            \PY{l+s+sd}{\PYZdq{}\PYZdq{}\PYZdq{}Face Landmarks dataset.\PYZdq{}\PYZdq{}\PYZdq{}}
        
            \PY{k}{def} \PY{n+nf}{\PYZus{}\PYZus{}init\PYZus{}\PYZus{}}\PY{p}{(}\PY{n+nb+bp}{self}\PY{p}{,} \PY{n}{csv\PYZus{}file}\PY{p}{,} \PY{n}{root\PYZus{}dir}\PY{p}{,} \PY{n}{transform}\PY{o}{=}\PY{k+kc}{None}\PY{p}{)}\PY{p}{:}
                \PY{l+s+sd}{\PYZdq{}\PYZdq{}\PYZdq{}}
        \PY{l+s+sd}{        Args:}
        \PY{l+s+sd}{            csv\PYZus{}file (string): Path to the csv file with annotations.}
        \PY{l+s+sd}{            root\PYZus{}dir (string): Directory with all the images.}
        \PY{l+s+sd}{            transform (callable, optional): Optional transform to be applied}
        \PY{l+s+sd}{                on a sample.}
        \PY{l+s+sd}{        \PYZdq{}\PYZdq{}\PYZdq{}}
                \PY{n+nb+bp}{self}\PY{o}{.}\PY{n}{key\PYZus{}pts\PYZus{}frame} \PY{o}{=} \PY{n}{pd}\PY{o}{.}\PY{n}{read\PYZus{}csv}\PY{p}{(}\PY{n}{csv\PYZus{}file}\PY{p}{)}
                \PY{n+nb+bp}{self}\PY{o}{.}\PY{n}{root\PYZus{}dir} \PY{o}{=} \PY{n}{root\PYZus{}dir}
                \PY{n+nb+bp}{self}\PY{o}{.}\PY{n}{transform} \PY{o}{=} \PY{n}{transform}
        
            \PY{k}{def} \PY{n+nf}{\PYZus{}\PYZus{}len\PYZus{}\PYZus{}}\PY{p}{(}\PY{n+nb+bp}{self}\PY{p}{)}\PY{p}{:}
                \PY{k}{return} \PY{n+nb}{len}\PY{p}{(}\PY{n+nb+bp}{self}\PY{o}{.}\PY{n}{key\PYZus{}pts\PYZus{}frame}\PY{p}{)}
        
            \PY{k}{def} \PY{n+nf}{\PYZus{}\PYZus{}getitem\PYZus{}\PYZus{}}\PY{p}{(}\PY{n+nb+bp}{self}\PY{p}{,} \PY{n}{idx}\PY{p}{)}\PY{p}{:}
                \PY{n}{image\PYZus{}name} \PY{o}{=} \PY{n}{os}\PY{o}{.}\PY{n}{path}\PY{o}{.}\PY{n}{join}\PY{p}{(}\PY{n+nb+bp}{self}\PY{o}{.}\PY{n}{root\PYZus{}dir}\PY{p}{,}
                                        \PY{n+nb+bp}{self}\PY{o}{.}\PY{n}{key\PYZus{}pts\PYZus{}frame}\PY{o}{.}\PY{n}{iloc}\PY{p}{[}\PY{n}{idx}\PY{p}{,} \PY{l+m+mi}{0}\PY{p}{]}\PY{p}{)}
                
                \PY{n}{image} \PY{o}{=} \PY{n}{mpimg}\PY{o}{.}\PY{n}{imread}\PY{p}{(}\PY{n}{image\PYZus{}name}\PY{p}{)}
                
                \PY{c+c1}{\PYZsh{} if image has an alpha color channel, get rid of it}
                \PY{k}{if}\PY{p}{(}\PY{n}{image}\PY{o}{.}\PY{n}{shape}\PY{p}{[}\PY{l+m+mi}{2}\PY{p}{]} \PY{o}{==} \PY{l+m+mi}{4}\PY{p}{)}\PY{p}{:}
                    \PY{n}{image} \PY{o}{=} \PY{n}{image}\PY{p}{[}\PY{p}{:}\PY{p}{,}\PY{p}{:}\PY{p}{,}\PY{l+m+mi}{0}\PY{p}{:}\PY{l+m+mi}{3}\PY{p}{]}
                
                \PY{n}{key\PYZus{}pts} \PY{o}{=} \PY{n+nb+bp}{self}\PY{o}{.}\PY{n}{key\PYZus{}pts\PYZus{}frame}\PY{o}{.}\PY{n}{iloc}\PY{p}{[}\PY{n}{idx}\PY{p}{,} \PY{l+m+mi}{1}\PY{p}{:}\PY{p}{]}\PY{o}{.}\PY{n}{as\PYZus{}matrix}\PY{p}{(}\PY{p}{)}
                \PY{n}{key\PYZus{}pts} \PY{o}{=} \PY{n}{key\PYZus{}pts}\PY{o}{.}\PY{n}{astype}\PY{p}{(}\PY{l+s+s1}{\PYZsq{}}\PY{l+s+s1}{float}\PY{l+s+s1}{\PYZsq{}}\PY{p}{)}\PY{o}{.}\PY{n}{reshape}\PY{p}{(}\PY{o}{\PYZhy{}}\PY{l+m+mi}{1}\PY{p}{,} \PY{l+m+mi}{2}\PY{p}{)}
                \PY{n}{sample} \PY{o}{=} \PY{p}{\PYZob{}}\PY{l+s+s1}{\PYZsq{}}\PY{l+s+s1}{image}\PY{l+s+s1}{\PYZsq{}}\PY{p}{:} \PY{n}{image}\PY{p}{,} \PY{l+s+s1}{\PYZsq{}}\PY{l+s+s1}{keypoints}\PY{l+s+s1}{\PYZsq{}}\PY{p}{:} \PY{n}{key\PYZus{}pts}\PY{p}{\PYZcb{}}
        
                \PY{k}{if} \PY{n+nb+bp}{self}\PY{o}{.}\PY{n}{transform}\PY{p}{:}
                    \PY{n}{sample} \PY{o}{=} \PY{n+nb+bp}{self}\PY{o}{.}\PY{n}{transform}\PY{p}{(}\PY{n}{sample}\PY{p}{)}
        
                \PY{k}{return} \PY{n}{sample}
\end{Verbatim}


    Now that we've defined this class, let's instantiate the dataset and
display some images.

    \begin{Verbatim}[commandchars=\\\{\}]
{\color{incolor}In [{\color{incolor}7}]:} \PY{c+c1}{\PYZsh{} Construct the dataset}
        \PY{n}{face\PYZus{}dataset} \PY{o}{=} \PY{n}{FacialKeypointsDataset}\PY{p}{(}\PY{n}{csv\PYZus{}file}\PY{o}{=}\PY{l+s+s1}{\PYZsq{}}\PY{l+s+s1}{data/training\PYZus{}frames\PYZus{}keypoints.csv}\PY{l+s+s1}{\PYZsq{}}\PY{p}{,}
                                              \PY{n}{root\PYZus{}dir}\PY{o}{=}\PY{l+s+s1}{\PYZsq{}}\PY{l+s+s1}{data/training/}\PY{l+s+s1}{\PYZsq{}}\PY{p}{)}
        
        \PY{c+c1}{\PYZsh{} print some stats about the dataset}
        \PY{n+nb}{print}\PY{p}{(}\PY{l+s+s1}{\PYZsq{}}\PY{l+s+s1}{Length of dataset: }\PY{l+s+s1}{\PYZsq{}}\PY{p}{,} \PY{n+nb}{len}\PY{p}{(}\PY{n}{face\PYZus{}dataset}\PY{p}{)}\PY{p}{)}
\end{Verbatim}


    \begin{Verbatim}[commandchars=\\\{\}]
Length of dataset:  3462

    \end{Verbatim}

    \begin{Verbatim}[commandchars=\\\{\}]
{\color{incolor}In [{\color{incolor}8}]:} \PY{c+c1}{\PYZsh{} Display a few of the images from the dataset}
        \PY{n}{num\PYZus{}to\PYZus{}display} \PY{o}{=} \PY{l+m+mi}{3}
        
        \PY{k}{for} \PY{n}{i} \PY{o+ow}{in} \PY{n+nb}{range}\PY{p}{(}\PY{n}{num\PYZus{}to\PYZus{}display}\PY{p}{)}\PY{p}{:}
            
            \PY{c+c1}{\PYZsh{} define the size of images}
            \PY{n}{fig} \PY{o}{=} \PY{n}{plt}\PY{o}{.}\PY{n}{figure}\PY{p}{(}\PY{n}{figsize}\PY{o}{=}\PY{p}{(}\PY{l+m+mi}{20}\PY{p}{,}\PY{l+m+mi}{10}\PY{p}{)}\PY{p}{)}
            
            \PY{c+c1}{\PYZsh{} randomly select a sample}
            \PY{n}{rand\PYZus{}i} \PY{o}{=} \PY{n}{np}\PY{o}{.}\PY{n}{random}\PY{o}{.}\PY{n}{randint}\PY{p}{(}\PY{l+m+mi}{0}\PY{p}{,} \PY{n+nb}{len}\PY{p}{(}\PY{n}{face\PYZus{}dataset}\PY{p}{)}\PY{p}{)}
            \PY{n}{sample} \PY{o}{=} \PY{n}{face\PYZus{}dataset}\PY{p}{[}\PY{n}{rand\PYZus{}i}\PY{p}{]}
        
            \PY{c+c1}{\PYZsh{} print the shape of the image and keypoints}
            \PY{n+nb}{print}\PY{p}{(}\PY{n}{i}\PY{p}{,} \PY{n}{sample}\PY{p}{[}\PY{l+s+s1}{\PYZsq{}}\PY{l+s+s1}{image}\PY{l+s+s1}{\PYZsq{}}\PY{p}{]}\PY{o}{.}\PY{n}{shape}\PY{p}{,} \PY{n}{sample}\PY{p}{[}\PY{l+s+s1}{\PYZsq{}}\PY{l+s+s1}{keypoints}\PY{l+s+s1}{\PYZsq{}}\PY{p}{]}\PY{o}{.}\PY{n}{shape}\PY{p}{)}
        
            \PY{n}{ax} \PY{o}{=} \PY{n}{plt}\PY{o}{.}\PY{n}{subplot}\PY{p}{(}\PY{l+m+mi}{1}\PY{p}{,} \PY{n}{num\PYZus{}to\PYZus{}display}\PY{p}{,} \PY{n}{i} \PY{o}{+} \PY{l+m+mi}{1}\PY{p}{)}
            \PY{n}{ax}\PY{o}{.}\PY{n}{set\PYZus{}title}\PY{p}{(}\PY{l+s+s1}{\PYZsq{}}\PY{l+s+s1}{Sample \PYZsh{}}\PY{l+s+si}{\PYZob{}\PYZcb{}}\PY{l+s+s1}{\PYZsq{}}\PY{o}{.}\PY{n}{format}\PY{p}{(}\PY{n}{i}\PY{p}{)}\PY{p}{)}
            
            \PY{c+c1}{\PYZsh{} Using the same display function, defined earlier}
            \PY{n}{show\PYZus{}keypoints}\PY{p}{(}\PY{n}{sample}\PY{p}{[}\PY{l+s+s1}{\PYZsq{}}\PY{l+s+s1}{image}\PY{l+s+s1}{\PYZsq{}}\PY{p}{]}\PY{p}{,} \PY{n}{sample}\PY{p}{[}\PY{l+s+s1}{\PYZsq{}}\PY{l+s+s1}{keypoints}\PY{l+s+s1}{\PYZsq{}}\PY{p}{]}\PY{p}{)}
\end{Verbatim}


    \begin{Verbatim}[commandchars=\\\{\}]
0 (164, 117, 3) (68, 2)
1 (230, 206, 3) (68, 2)
2 (299, 275, 3) (68, 2)

    \end{Verbatim}

    \begin{center}
    \adjustimage{max size={0.9\linewidth}{0.9\paperheight}}{output_12_1.png}
    \end{center}
    { \hspace*{\fill} \\}
    
    \begin{center}
    \adjustimage{max size={0.9\linewidth}{0.9\paperheight}}{output_12_2.png}
    \end{center}
    { \hspace*{\fill} \\}
    
    \begin{center}
    \adjustimage{max size={0.9\linewidth}{0.9\paperheight}}{output_12_3.png}
    \end{center}
    { \hspace*{\fill} \\}
    
    \subsection{Transforms}\label{transforms}

Now, the images above are not of the same size, and neural networks
often expect images that are standardized; a fixed size, with a
normalized range for color ranges and coordinates, and (for PyTorch)
converted from numpy lists and arrays to Tensors.

Therefore, we will need to write some pre-processing code. Let's create
four transforms:

\begin{itemize}
\tightlist
\item
  \texttt{Normalize}: to convert a color image to grayscale values with
  a range of {[}0,1{]} and normalize the keypoints to be in a range of
  about {[}-1, 1{]}
\item
  \texttt{Rescale}: to rescale an image to a desired size.
\item
  \texttt{RandomCrop}: to crop an image randomly.
\item
  \texttt{ToTensor}: to convert numpy images to torch images.
\end{itemize}

We will write them as callable classes instead of simple functions so
that parameters of the transform need not be passed everytime it's
called. For this, we just need to implement \texttt{\_\_call\_\_} method
and (if we require parameters to be passed in), the
\texttt{\_\_init\_\_} method. We can then use a transform like this:

\begin{verbatim}
tx = Transform(params)
transformed_sample = tx(sample)
\end{verbatim}

Observe below how these transforms are generally applied to both the
image and its keypoints.

    \begin{Verbatim}[commandchars=\\\{\}]
{\color{incolor}In [{\color{incolor}9}]:} \PY{k+kn}{import} \PY{n+nn}{torch}
        \PY{k+kn}{from} \PY{n+nn}{torchvision} \PY{k}{import} \PY{n}{transforms}\PY{p}{,} \PY{n}{utils}
        \PY{c+c1}{\PYZsh{} tranforms}
        
        \PY{k}{class} \PY{n+nc}{Normalize}\PY{p}{(}\PY{n+nb}{object}\PY{p}{)}\PY{p}{:}
            \PY{l+s+sd}{\PYZdq{}\PYZdq{}\PYZdq{}Convert a color image to grayscale and normalize the color range to [0,1].\PYZdq{}\PYZdq{}\PYZdq{}}        
        
            \PY{k}{def} \PY{n+nf}{\PYZus{}\PYZus{}call\PYZus{}\PYZus{}}\PY{p}{(}\PY{n+nb+bp}{self}\PY{p}{,} \PY{n}{sample}\PY{p}{)}\PY{p}{:}
                \PY{n}{image}\PY{p}{,} \PY{n}{key\PYZus{}pts} \PY{o}{=} \PY{n}{sample}\PY{p}{[}\PY{l+s+s1}{\PYZsq{}}\PY{l+s+s1}{image}\PY{l+s+s1}{\PYZsq{}}\PY{p}{]}\PY{p}{,} \PY{n}{sample}\PY{p}{[}\PY{l+s+s1}{\PYZsq{}}\PY{l+s+s1}{keypoints}\PY{l+s+s1}{\PYZsq{}}\PY{p}{]}
                
                \PY{n}{image\PYZus{}copy} \PY{o}{=} \PY{n}{np}\PY{o}{.}\PY{n}{copy}\PY{p}{(}\PY{n}{image}\PY{p}{)}
                \PY{n}{key\PYZus{}pts\PYZus{}copy} \PY{o}{=} \PY{n}{np}\PY{o}{.}\PY{n}{copy}\PY{p}{(}\PY{n}{key\PYZus{}pts}\PY{p}{)}
        
                \PY{c+c1}{\PYZsh{} convert image to grayscale}
                \PY{n}{image\PYZus{}copy} \PY{o}{=} \PY{n}{cv2}\PY{o}{.}\PY{n}{cvtColor}\PY{p}{(}\PY{n}{image}\PY{p}{,} \PY{n}{cv2}\PY{o}{.}\PY{n}{COLOR\PYZus{}RGB2GRAY}\PY{p}{)}
                
                \PY{c+c1}{\PYZsh{} scale color range from [0, 255] to [0, 1]}
                \PY{n}{image\PYZus{}copy}\PY{o}{=}  \PY{n}{image\PYZus{}copy}\PY{o}{/}\PY{l+m+mf}{255.0}
                
                \PY{c+c1}{\PYZsh{} scale keypoints to be centered around 0 with a range of [\PYZhy{}1, 1]}
                \PY{c+c1}{\PYZsh{} mean = 100, sqrt = 50, so, pts should be (pts \PYZhy{} 100)/50}
                \PY{n}{key\PYZus{}pts\PYZus{}copy} \PY{o}{=} \PY{p}{(}\PY{n}{key\PYZus{}pts\PYZus{}copy} \PY{o}{\PYZhy{}} \PY{l+m+mi}{100}\PY{p}{)}\PY{o}{/}\PY{l+m+mf}{50.0}
        
        
                \PY{k}{return} \PY{p}{\PYZob{}}\PY{l+s+s1}{\PYZsq{}}\PY{l+s+s1}{image}\PY{l+s+s1}{\PYZsq{}}\PY{p}{:} \PY{n}{image\PYZus{}copy}\PY{p}{,} \PY{l+s+s1}{\PYZsq{}}\PY{l+s+s1}{keypoints}\PY{l+s+s1}{\PYZsq{}}\PY{p}{:} \PY{n}{key\PYZus{}pts\PYZus{}copy}\PY{p}{\PYZcb{}}
        
        
        \PY{k}{class} \PY{n+nc}{Rescale}\PY{p}{(}\PY{n+nb}{object}\PY{p}{)}\PY{p}{:}
            \PY{l+s+sd}{\PYZdq{}\PYZdq{}\PYZdq{}Rescale the image in a sample to a given size.}
        
        \PY{l+s+sd}{    Args:}
        \PY{l+s+sd}{        output\PYZus{}size (tuple or int): Desired output size. If tuple, output is}
        \PY{l+s+sd}{            matched to output\PYZus{}size. If int, smaller of image edges is matched}
        \PY{l+s+sd}{            to output\PYZus{}size keeping aspect ratio the same.}
        \PY{l+s+sd}{    \PYZdq{}\PYZdq{}\PYZdq{}}
        
            \PY{k}{def} \PY{n+nf}{\PYZus{}\PYZus{}init\PYZus{}\PYZus{}}\PY{p}{(}\PY{n+nb+bp}{self}\PY{p}{,} \PY{n}{output\PYZus{}size}\PY{p}{)}\PY{p}{:}
                \PY{k}{assert} \PY{n+nb}{isinstance}\PY{p}{(}\PY{n}{output\PYZus{}size}\PY{p}{,} \PY{p}{(}\PY{n+nb}{int}\PY{p}{,} \PY{n+nb}{tuple}\PY{p}{)}\PY{p}{)}
                \PY{n+nb+bp}{self}\PY{o}{.}\PY{n}{output\PYZus{}size} \PY{o}{=} \PY{n}{output\PYZus{}size}
        
            \PY{k}{def} \PY{n+nf}{\PYZus{}\PYZus{}call\PYZus{}\PYZus{}}\PY{p}{(}\PY{n+nb+bp}{self}\PY{p}{,} \PY{n}{sample}\PY{p}{)}\PY{p}{:}
                \PY{n}{image}\PY{p}{,} \PY{n}{key\PYZus{}pts} \PY{o}{=} \PY{n}{sample}\PY{p}{[}\PY{l+s+s1}{\PYZsq{}}\PY{l+s+s1}{image}\PY{l+s+s1}{\PYZsq{}}\PY{p}{]}\PY{p}{,} \PY{n}{sample}\PY{p}{[}\PY{l+s+s1}{\PYZsq{}}\PY{l+s+s1}{keypoints}\PY{l+s+s1}{\PYZsq{}}\PY{p}{]}
        
                \PY{n}{h}\PY{p}{,} \PY{n}{w} \PY{o}{=} \PY{n}{image}\PY{o}{.}\PY{n}{shape}\PY{p}{[}\PY{p}{:}\PY{l+m+mi}{2}\PY{p}{]}
                \PY{k}{if} \PY{n+nb}{isinstance}\PY{p}{(}\PY{n+nb+bp}{self}\PY{o}{.}\PY{n}{output\PYZus{}size}\PY{p}{,} \PY{n+nb}{int}\PY{p}{)}\PY{p}{:}
                    \PY{k}{if} \PY{n}{h} \PY{o}{\PYZgt{}} \PY{n}{w}\PY{p}{:}
                        \PY{n}{new\PYZus{}h}\PY{p}{,} \PY{n}{new\PYZus{}w} \PY{o}{=} \PY{n+nb+bp}{self}\PY{o}{.}\PY{n}{output\PYZus{}size} \PY{o}{*} \PY{n}{h} \PY{o}{/} \PY{n}{w}\PY{p}{,} \PY{n+nb+bp}{self}\PY{o}{.}\PY{n}{output\PYZus{}size}
                    \PY{k}{else}\PY{p}{:}
                        \PY{n}{new\PYZus{}h}\PY{p}{,} \PY{n}{new\PYZus{}w} \PY{o}{=} \PY{n+nb+bp}{self}\PY{o}{.}\PY{n}{output\PYZus{}size}\PY{p}{,} \PY{n+nb+bp}{self}\PY{o}{.}\PY{n}{output\PYZus{}size} \PY{o}{*} \PY{n}{w} \PY{o}{/} \PY{n}{h}
                \PY{k}{else}\PY{p}{:}
                    \PY{n}{new\PYZus{}h}\PY{p}{,} \PY{n}{new\PYZus{}w} \PY{o}{=} \PY{n+nb+bp}{self}\PY{o}{.}\PY{n}{output\PYZus{}size}
        
                \PY{n}{new\PYZus{}h}\PY{p}{,} \PY{n}{new\PYZus{}w} \PY{o}{=} \PY{n+nb}{int}\PY{p}{(}\PY{n}{new\PYZus{}h}\PY{p}{)}\PY{p}{,} \PY{n+nb}{int}\PY{p}{(}\PY{n}{new\PYZus{}w}\PY{p}{)}
        
                \PY{n}{img} \PY{o}{=} \PY{n}{cv2}\PY{o}{.}\PY{n}{resize}\PY{p}{(}\PY{n}{image}\PY{p}{,} \PY{p}{(}\PY{n}{new\PYZus{}w}\PY{p}{,} \PY{n}{new\PYZus{}h}\PY{p}{)}\PY{p}{)}
                
                \PY{c+c1}{\PYZsh{} scale the pts, too}
                \PY{n}{key\PYZus{}pts} \PY{o}{=} \PY{n}{key\PYZus{}pts} \PY{o}{*} \PY{p}{[}\PY{n}{new\PYZus{}w} \PY{o}{/} \PY{n}{w}\PY{p}{,} \PY{n}{new\PYZus{}h} \PY{o}{/} \PY{n}{h}\PY{p}{]}
        
                \PY{k}{return} \PY{p}{\PYZob{}}\PY{l+s+s1}{\PYZsq{}}\PY{l+s+s1}{image}\PY{l+s+s1}{\PYZsq{}}\PY{p}{:} \PY{n}{img}\PY{p}{,} \PY{l+s+s1}{\PYZsq{}}\PY{l+s+s1}{keypoints}\PY{l+s+s1}{\PYZsq{}}\PY{p}{:} \PY{n}{key\PYZus{}pts}\PY{p}{\PYZcb{}}
        
        
        \PY{k}{class} \PY{n+nc}{RandomCrop}\PY{p}{(}\PY{n+nb}{object}\PY{p}{)}\PY{p}{:}
            \PY{l+s+sd}{\PYZdq{}\PYZdq{}\PYZdq{}Crop randomly the image in a sample.}
        
        \PY{l+s+sd}{    Args:}
        \PY{l+s+sd}{        output\PYZus{}size (tuple or int): Desired output size. If int, square crop}
        \PY{l+s+sd}{            is made.}
        \PY{l+s+sd}{    \PYZdq{}\PYZdq{}\PYZdq{}}
        
            \PY{k}{def} \PY{n+nf}{\PYZus{}\PYZus{}init\PYZus{}\PYZus{}}\PY{p}{(}\PY{n+nb+bp}{self}\PY{p}{,} \PY{n}{output\PYZus{}size}\PY{p}{)}\PY{p}{:}
                \PY{k}{assert} \PY{n+nb}{isinstance}\PY{p}{(}\PY{n}{output\PYZus{}size}\PY{p}{,} \PY{p}{(}\PY{n+nb}{int}\PY{p}{,} \PY{n+nb}{tuple}\PY{p}{)}\PY{p}{)}
                \PY{k}{if} \PY{n+nb}{isinstance}\PY{p}{(}\PY{n}{output\PYZus{}size}\PY{p}{,} \PY{n+nb}{int}\PY{p}{)}\PY{p}{:}
                    \PY{n+nb+bp}{self}\PY{o}{.}\PY{n}{output\PYZus{}size} \PY{o}{=} \PY{p}{(}\PY{n}{output\PYZus{}size}\PY{p}{,} \PY{n}{output\PYZus{}size}\PY{p}{)}
                \PY{k}{else}\PY{p}{:}
                    \PY{k}{assert} \PY{n+nb}{len}\PY{p}{(}\PY{n}{output\PYZus{}size}\PY{p}{)} \PY{o}{==} \PY{l+m+mi}{2}
                    \PY{n+nb+bp}{self}\PY{o}{.}\PY{n}{output\PYZus{}size} \PY{o}{=} \PY{n}{output\PYZus{}size}
        
            \PY{k}{def} \PY{n+nf}{\PYZus{}\PYZus{}call\PYZus{}\PYZus{}}\PY{p}{(}\PY{n+nb+bp}{self}\PY{p}{,} \PY{n}{sample}\PY{p}{)}\PY{p}{:}
                \PY{n}{image}\PY{p}{,} \PY{n}{key\PYZus{}pts} \PY{o}{=} \PY{n}{sample}\PY{p}{[}\PY{l+s+s1}{\PYZsq{}}\PY{l+s+s1}{image}\PY{l+s+s1}{\PYZsq{}}\PY{p}{]}\PY{p}{,} \PY{n}{sample}\PY{p}{[}\PY{l+s+s1}{\PYZsq{}}\PY{l+s+s1}{keypoints}\PY{l+s+s1}{\PYZsq{}}\PY{p}{]}
        
                \PY{n}{h}\PY{p}{,} \PY{n}{w} \PY{o}{=} \PY{n}{image}\PY{o}{.}\PY{n}{shape}\PY{p}{[}\PY{p}{:}\PY{l+m+mi}{2}\PY{p}{]}
                \PY{n}{new\PYZus{}h}\PY{p}{,} \PY{n}{new\PYZus{}w} \PY{o}{=} \PY{n+nb+bp}{self}\PY{o}{.}\PY{n}{output\PYZus{}size}
        
                \PY{n}{top} \PY{o}{=} \PY{n}{np}\PY{o}{.}\PY{n}{random}\PY{o}{.}\PY{n}{randint}\PY{p}{(}\PY{l+m+mi}{0}\PY{p}{,} \PY{n}{h} \PY{o}{\PYZhy{}} \PY{n}{new\PYZus{}h}\PY{p}{)}
                \PY{n}{left} \PY{o}{=} \PY{n}{np}\PY{o}{.}\PY{n}{random}\PY{o}{.}\PY{n}{randint}\PY{p}{(}\PY{l+m+mi}{0}\PY{p}{,} \PY{n}{w} \PY{o}{\PYZhy{}} \PY{n}{new\PYZus{}w}\PY{p}{)}
        
                \PY{n}{image} \PY{o}{=} \PY{n}{image}\PY{p}{[}\PY{n}{top}\PY{p}{:} \PY{n}{top} \PY{o}{+} \PY{n}{new\PYZus{}h}\PY{p}{,}
                              \PY{n}{left}\PY{p}{:} \PY{n}{left} \PY{o}{+} \PY{n}{new\PYZus{}w}\PY{p}{]}
        
                \PY{n}{key\PYZus{}pts} \PY{o}{=} \PY{n}{key\PYZus{}pts} \PY{o}{\PYZhy{}} \PY{p}{[}\PY{n}{left}\PY{p}{,} \PY{n}{top}\PY{p}{]}
        
                \PY{k}{return} \PY{p}{\PYZob{}}\PY{l+s+s1}{\PYZsq{}}\PY{l+s+s1}{image}\PY{l+s+s1}{\PYZsq{}}\PY{p}{:} \PY{n}{image}\PY{p}{,} \PY{l+s+s1}{\PYZsq{}}\PY{l+s+s1}{keypoints}\PY{l+s+s1}{\PYZsq{}}\PY{p}{:} \PY{n}{key\PYZus{}pts}\PY{p}{\PYZcb{}}
        
        
        \PY{k}{class} \PY{n+nc}{ToTensor}\PY{p}{(}\PY{n+nb}{object}\PY{p}{)}\PY{p}{:}
            \PY{l+s+sd}{\PYZdq{}\PYZdq{}\PYZdq{}Convert ndarrays in sample to Tensors.\PYZdq{}\PYZdq{}\PYZdq{}}
        
            \PY{k}{def} \PY{n+nf}{\PYZus{}\PYZus{}call\PYZus{}\PYZus{}}\PY{p}{(}\PY{n+nb+bp}{self}\PY{p}{,} \PY{n}{sample}\PY{p}{)}\PY{p}{:}
                \PY{n}{image}\PY{p}{,} \PY{n}{key\PYZus{}pts} \PY{o}{=} \PY{n}{sample}\PY{p}{[}\PY{l+s+s1}{\PYZsq{}}\PY{l+s+s1}{image}\PY{l+s+s1}{\PYZsq{}}\PY{p}{]}\PY{p}{,} \PY{n}{sample}\PY{p}{[}\PY{l+s+s1}{\PYZsq{}}\PY{l+s+s1}{keypoints}\PY{l+s+s1}{\PYZsq{}}\PY{p}{]}
                 
                \PY{c+c1}{\PYZsh{} if image has no grayscale color channel, add one}
                \PY{k}{if}\PY{p}{(}\PY{n+nb}{len}\PY{p}{(}\PY{n}{image}\PY{o}{.}\PY{n}{shape}\PY{p}{)} \PY{o}{==} \PY{l+m+mi}{2}\PY{p}{)}\PY{p}{:}
                    \PY{c+c1}{\PYZsh{} add that third color dim}
                    \PY{n}{image} \PY{o}{=} \PY{n}{image}\PY{o}{.}\PY{n}{reshape}\PY{p}{(}\PY{n}{image}\PY{o}{.}\PY{n}{shape}\PY{p}{[}\PY{l+m+mi}{0}\PY{p}{]}\PY{p}{,} \PY{n}{image}\PY{o}{.}\PY{n}{shape}\PY{p}{[}\PY{l+m+mi}{1}\PY{p}{]}\PY{p}{,} \PY{l+m+mi}{1}\PY{p}{)}
                    
                \PY{c+c1}{\PYZsh{} swap color axis because}
                \PY{c+c1}{\PYZsh{} numpy image: H x W x C}
                \PY{c+c1}{\PYZsh{} torch image: C X H X W}
                \PY{n}{image} \PY{o}{=} \PY{n}{image}\PY{o}{.}\PY{n}{transpose}\PY{p}{(}\PY{p}{(}\PY{l+m+mi}{2}\PY{p}{,} \PY{l+m+mi}{0}\PY{p}{,} \PY{l+m+mi}{1}\PY{p}{)}\PY{p}{)}
                
                \PY{k}{return} \PY{p}{\PYZob{}}\PY{l+s+s1}{\PYZsq{}}\PY{l+s+s1}{image}\PY{l+s+s1}{\PYZsq{}}\PY{p}{:} \PY{n}{torch}\PY{o}{.}\PY{n}{from\PYZus{}numpy}\PY{p}{(}\PY{n}{image}\PY{p}{)}\PY{p}{,}
                        \PY{l+s+s1}{\PYZsq{}}\PY{l+s+s1}{keypoints}\PY{l+s+s1}{\PYZsq{}}\PY{p}{:} \PY{n}{torch}\PY{o}{.}\PY{n}{from\PYZus{}numpy}\PY{p}{(}\PY{n}{key\PYZus{}pts}\PY{p}{)}\PY{p}{\PYZcb{}}
\end{Verbatim}


    \subsection{Test out the transforms}\label{test-out-the-transforms}

Let's test these transforms out to make sure they behave as expected. As
you look at each transform, note that, in this case, \textbf{order does
matter}. For example, you cannot crop a image using a value smaller than
the original image (and the orginal images vary in size!), but, if you
first rescale the original image, you can then crop it to any size
smaller than the rescaled size.

    \begin{Verbatim}[commandchars=\\\{\}]
{\color{incolor}In [{\color{incolor}10}]:} \PY{c+c1}{\PYZsh{} test out some of these transforms}
         \PY{n}{rescale} \PY{o}{=} \PY{n}{Rescale}\PY{p}{(}\PY{l+m+mi}{100}\PY{p}{)}
         \PY{n}{crop} \PY{o}{=} \PY{n}{RandomCrop}\PY{p}{(}\PY{l+m+mi}{50}\PY{p}{)}
         \PY{n}{composed} \PY{o}{=} \PY{n}{transforms}\PY{o}{.}\PY{n}{Compose}\PY{p}{(}\PY{p}{[}\PY{n}{Rescale}\PY{p}{(}\PY{l+m+mi}{250}\PY{p}{)}\PY{p}{,}
                                        \PY{n}{RandomCrop}\PY{p}{(}\PY{l+m+mi}{224}\PY{p}{)}\PY{p}{]}\PY{p}{)}
         
         \PY{c+c1}{\PYZsh{} apply the transforms to a sample image}
         \PY{n}{test\PYZus{}num} \PY{o}{=} \PY{l+m+mi}{500}
         \PY{n}{sample} \PY{o}{=} \PY{n}{face\PYZus{}dataset}\PY{p}{[}\PY{n}{test\PYZus{}num}\PY{p}{]}
         
         \PY{n}{fig} \PY{o}{=} \PY{n}{plt}\PY{o}{.}\PY{n}{figure}\PY{p}{(}\PY{p}{)}
         \PY{k}{for} \PY{n}{i}\PY{p}{,} \PY{n}{tx} \PY{o+ow}{in} \PY{n+nb}{enumerate}\PY{p}{(}\PY{p}{[}\PY{n}{rescale}\PY{p}{,} \PY{n}{crop}\PY{p}{,} \PY{n}{composed}\PY{p}{]}\PY{p}{)}\PY{p}{:}
             \PY{n}{transformed\PYZus{}sample} \PY{o}{=} \PY{n}{tx}\PY{p}{(}\PY{n}{sample}\PY{p}{)}
         
             \PY{n}{ax} \PY{o}{=} \PY{n}{plt}\PY{o}{.}\PY{n}{subplot}\PY{p}{(}\PY{l+m+mi}{1}\PY{p}{,} \PY{l+m+mi}{3}\PY{p}{,} \PY{n}{i} \PY{o}{+} \PY{l+m+mi}{1}\PY{p}{)}
             \PY{n}{plt}\PY{o}{.}\PY{n}{tight\PYZus{}layout}\PY{p}{(}\PY{p}{)}
             \PY{n}{ax}\PY{o}{.}\PY{n}{set\PYZus{}title}\PY{p}{(}\PY{n+nb}{type}\PY{p}{(}\PY{n}{tx}\PY{p}{)}\PY{o}{.}\PY{n+nv+vm}{\PYZus{}\PYZus{}name\PYZus{}\PYZus{}}\PY{p}{)}
             \PY{n}{show\PYZus{}keypoints}\PY{p}{(}\PY{n}{transformed\PYZus{}sample}\PY{p}{[}\PY{l+s+s1}{\PYZsq{}}\PY{l+s+s1}{image}\PY{l+s+s1}{\PYZsq{}}\PY{p}{]}\PY{p}{,} \PY{n}{transformed\PYZus{}sample}\PY{p}{[}\PY{l+s+s1}{\PYZsq{}}\PY{l+s+s1}{keypoints}\PY{l+s+s1}{\PYZsq{}}\PY{p}{]}\PY{p}{)}
         
         \PY{n}{plt}\PY{o}{.}\PY{n}{show}\PY{p}{(}\PY{p}{)}
\end{Verbatim}


    \begin{center}
    \adjustimage{max size={0.9\linewidth}{0.9\paperheight}}{output_16_0.png}
    \end{center}
    { \hspace*{\fill} \\}
    
    \subsection{Create the transformed
dataset}\label{create-the-transformed-dataset}

Apply the transforms in order to get grayscale images of the same shape.
Verify that your transform works by printing out the shape of the
resulting data (printing out a few examples should show you a consistent
tensor size).

    \begin{Verbatim}[commandchars=\\\{\}]
{\color{incolor}In [{\color{incolor}11}]:} \PY{c+c1}{\PYZsh{} define the data tranform}
         \PY{c+c1}{\PYZsh{} order matters! i.e. rescaling should come before a smaller crop}
         \PY{n}{data\PYZus{}transform} \PY{o}{=} \PY{n}{transforms}\PY{o}{.}\PY{n}{Compose}\PY{p}{(}\PY{p}{[}\PY{n}{Rescale}\PY{p}{(}\PY{l+m+mi}{250}\PY{p}{)}\PY{p}{,}
                                              \PY{n}{RandomCrop}\PY{p}{(}\PY{l+m+mi}{224}\PY{p}{)}\PY{p}{,}
                                              \PY{n}{Normalize}\PY{p}{(}\PY{p}{)}\PY{p}{,}
                                              \PY{n}{ToTensor}\PY{p}{(}\PY{p}{)}\PY{p}{]}\PY{p}{)}
         
         \PY{c+c1}{\PYZsh{} create the transformed dataset}
         \PY{n}{transformed\PYZus{}dataset} \PY{o}{=} \PY{n}{FacialKeypointsDataset}\PY{p}{(}\PY{n}{csv\PYZus{}file}\PY{o}{=}\PY{l+s+s1}{\PYZsq{}}\PY{l+s+s1}{data/training\PYZus{}frames\PYZus{}keypoints.csv}\PY{l+s+s1}{\PYZsq{}}\PY{p}{,}
                                                      \PY{n}{root\PYZus{}dir}\PY{o}{=}\PY{l+s+s1}{\PYZsq{}}\PY{l+s+s1}{data/training/}\PY{l+s+s1}{\PYZsq{}}\PY{p}{,}
                                                      \PY{n}{transform}\PY{o}{=}\PY{n}{data\PYZus{}transform}\PY{p}{)}
\end{Verbatim}


    \begin{Verbatim}[commandchars=\\\{\}]
{\color{incolor}In [{\color{incolor}12}]:} \PY{c+c1}{\PYZsh{} print some stats about the transformed data}
         \PY{n+nb}{print}\PY{p}{(}\PY{l+s+s1}{\PYZsq{}}\PY{l+s+s1}{Number of images: }\PY{l+s+s1}{\PYZsq{}}\PY{p}{,} \PY{n+nb}{len}\PY{p}{(}\PY{n}{transformed\PYZus{}dataset}\PY{p}{)}\PY{p}{)}
         
         \PY{c+c1}{\PYZsh{} make sure the sample tensors are the expected size}
         \PY{k}{for} \PY{n}{i} \PY{o+ow}{in} \PY{n+nb}{range}\PY{p}{(}\PY{l+m+mi}{5}\PY{p}{)}\PY{p}{:}
             \PY{n}{sample} \PY{o}{=} \PY{n}{transformed\PYZus{}dataset}\PY{p}{[}\PY{n}{i}\PY{p}{]}
             \PY{n+nb}{print}\PY{p}{(}\PY{n}{i}\PY{p}{,} \PY{n}{sample}\PY{p}{[}\PY{l+s+s1}{\PYZsq{}}\PY{l+s+s1}{image}\PY{l+s+s1}{\PYZsq{}}\PY{p}{]}\PY{o}{.}\PY{n}{size}\PY{p}{(}\PY{p}{)}\PY{p}{,} \PY{n}{sample}\PY{p}{[}\PY{l+s+s1}{\PYZsq{}}\PY{l+s+s1}{keypoints}\PY{l+s+s1}{\PYZsq{}}\PY{p}{]}\PY{o}{.}\PY{n}{size}\PY{p}{(}\PY{p}{)}\PY{p}{)}
\end{Verbatim}


    \begin{Verbatim}[commandchars=\\\{\}]
Number of images:  3462
0 torch.Size([1, 224, 224]) torch.Size([68, 2])
1 torch.Size([1, 224, 224]) torch.Size([68, 2])
2 torch.Size([1, 224, 224]) torch.Size([68, 2])
3 torch.Size([1, 224, 224]) torch.Size([68, 2])
4 torch.Size([1, 224, 224]) torch.Size([68, 2])

    \end{Verbatim}

    \subsection{Data Iteration and
Batching}\label{data-iteration-and-batching}

Right now, we are iterating over this data using a \texttt{for} loop,
but we are missing out on a lot of PyTorch's dataset capabilities,
specifically the abilities to:

\begin{itemize}
\tightlist
\item
  Batch the data
\item
  Shuffle the data
\item
  Load the data in parallel using \texttt{multiprocessing} workers.
\end{itemize}

\texttt{torch.utils.data.DataLoader} is an iterator which provides all
these features, and we'll see this in use in the \emph{next} notebook,
Notebook 2, when we load data in batches to train a neural network!

\begin{center}\rule{0.5\linewidth}{\linethickness}\end{center}

    \subsection{Ready to Train!}\label{ready-to-train}

Now that you've seen how to load and transform our data, you're ready to
build a neural network to train on this data.

In the next notebook, you'll be tasked with creating a CNN for facial
keypoint detection.


    % Add a bibliography block to the postdoc
    
    
    
    \end{document}
